\documentclass[10pt,a4paper,openright]{book}

%% Formateo del título del documento
\title{\Huge EJERCICIOS ANÁLISIS DE VARIABLE REAL}
\author{Juan Diego Barrado Daganzo} %\\ es salto de linea
\date{\today\footnote{Este documento se actualiza, para consultar las últimas versiones entrar en el enlace \url{https://github.com/JuanDiegoBarrado/???}}}

%% Formateo del estilo de escritura y de la pagina
\pagestyle{plain}
\setlength{\parskip}{0.35cm} %edicion de espaciado
\setlength{\parindent}{0cm} %edicion de sangría
\clubpenalty=10000 %líneas viudas NO
\widowpenalty=10000 %líneas viudas NO
\usepackage[top=2.5cm, bottom=2.5cm, left=3cm, right=3cm]{geometry} % para establecer las medidas de los margenes
\usepackage[spanish]{babel} %Para que el idioma por defecto sea español
\usepackage{ulem} % para poder subrayar entornos especiales como las secciones

%% Texto matematico y simbolos especiales
\usepackage{amsmath} %Paquetes para mates
\usepackage{amsfonts} %Paquetes para mates
\usepackage{amssymb} %Paquetes para mates
\usepackage{stmaryrd} % paquete para mates
\usepackage{latexsym} %Paquetes para mates
\usepackage{cancel} %Paquete tachar cosas

%% Ruta de las fotos e inclusion de las mismas
\usepackage{graphicx}
\graphicspath{{./fotos/}}

%% Paquete para subfigures dentro del entorno figure
\usepackage{subcaption}

%% Inclusion de referencias cruzadas por defecto y específicas
\usepackage[colorlinks=true]{hyperref}
\hypersetup{
	urlcolor=red,
	linkcolor=blue,
}

%% Paquete para definir y utilizar colores por el documento
\usepackage[dvipsnames,usenames]{xcolor} %activar e incluir colores
	%% definicion de los colores que se van a utilizar en cada cabecera
    \definecolor{capitulos}{RGB}{60,0,0}% gama de colores de los capitulos
    \definecolor{secciones}{RGB}{95,8,5}% gama de colores de las secciones
    \definecolor{subsecciones}{RGB}{140,36,31}% gama de colores de las subsections
    \definecolor{subsubsecciones}{RGB}{188,109,79}% gama de colores de las subsubsections
    \definecolor{teoremas}{RGB}{164,56,32}% gama de colores para los teoremas
    \definecolor{demos}{RGB}{105,105,105} % gama de colores para el cuerpo de las demostraciones

%% Paquete para la edición y el formateo de capítulos, secciones...
\usepackage[explicit]{titlesec}
	%% Definición del estilo de los capítulos, secciones, etc...
    \titleformat{\chapter}[display]{\normalfont\huge\bfseries\color{capitulos}}{}{0pt}{\Huge \uppercase{#1}}[\titlerule]
    \titleformat{\section}{\normalfont\Large\bfseries\color{secciones}}{}{0pt}{\uppercase{#1}}
    \titleformat{\subsection}{\normalfont\large\bfseries\color{subsecciones}}{}{0pt}{\uline{#1}}
    \titleformat{\subsubsection}{\normalfont\normalsize\bfseries\color{subsubsecciones}}{}{0pt}{#1}

%% Paquete para el formateo de entornos del proyecto
\newtheorem{ej}{Ejercicio}
\newtheorem{sol}{Solución Ejercicio}

%% Definicion de operadores especiales para simplificar la escritura matematica
\DeclareMathOperator{\dom}{dom}
\DeclareMathOperator{\img}{img}
\DeclareMathOperator{\rot}{rot}
\DeclareMathOperator{\divg}{div}
\newcommand{\dif}[1]{\ d#1}

%% Paquete e instrucciones para la generacion de los dibujos
\usepackage{pgfplots}
\pgfplotsset{compat=1.17}
\usepackage{tkz-fct}
\usepackage{pstricks}
\usepackage{pstcol} 
\usepackage{pst-node}
\usepackage{pst-plot}

%% Paquete para la tachar cosas
\usepackage{centernot}

%% Paquete para tener un apéndice
\usepackage{appendix}

%% Paquete para comentar bloques de código de LaTeX
\usepackage{verbatim}

\begin{document}
\maketitle
\frontmatter
\section*{QUIÉNES SOMOS}
Somos un grupo de estudiantes de la Universidad Complutense de Madrid, concretamente del Doble Grado de Informática y Matemáticas que queremos compartir unos apuntes de calidad y, como mínimo, ordenados para que os sea más fácil llevar la asignatura al día (sobre todo a estudiantes de Doble Grado).

Estos apuntes son posibles gracias a la colaboración de más alumnos como tú que deciden aportar un granito de arena al proyecto. Puedes contribuir de la siguiente manera:
\begin{itemize}
\item Notificando erratas
\item Modificando erratas
\item Proponiendo mejoras
\item Aportando ejemplos nuevos
\item Aportando nuevas versiones
\end{itemize}
Para contribuir no tienes más que ponerte en contacto con \href{mailto:juandbar@ucm.es}{juandbar@ucm.es} o dejarnos un \textit{Pull Request} en \url{https://github.com/JuanDiegoBarrado/CalculoDiferencial}. Los detalles para que la contribución de todos sea lo más homogénea posible estarán en el fichero \textit{Contribute.md} de dicho repositorio o, en caso de no aparecer correctamente, podéis poneros en contacto con el correo anteriormente mencionado.

Muchas gracias, esperamos que este documento te sea útil.

\section*{AGRADECIMIENTOS}
Queremos dar gracias especialmente al Profesor Javier Soria, por ser el profesor que impartió la asignatura de \textit{Cálculo Diferencial} durante la elaboración de estos apuntes y por darnos \textit{feedback} sobre la calidad y las posibles mejoras de los mismos.

También queremos dar las gracias a los Profesores Víctor Manuel Sánchez, Jose María Martínez Ansemil y Socorro Ponte Miramontes, por elaborar otros manuales más formales sobre la asignatura que nos han permitido contrastar adecuadamente los nuestros y entender adecuadamente los contenidos aquí expuestos.
\vfill
Cálculo Diferencial © 2021 by Juan Diego Barrado \& Iker Muñoz is licensed under Attribution-NonCommercial 4.0 International. To view a copy of this license, visit
\begin{center}
\url{http://creativecommons.org/licenses/by-nc/4.0/}
\end{center}

\mainmatter
\hypersetup{linkcolor=black} %% Conseguimos que el índice salga en negro
\setcounter{tocdepth}{3}% para que salgan las subsubsecciones en el indice
\setcounter{secnumdepth}{4}% para que salgan los números de las subsubsecciones en el indice
\tableofcontents
\hypersetup{linkcolor=blue} % retornamos el color de las referencias a azul, para que quede resaltado

\chapter{Conjuntos y Funciones}
\begin{ej}
Sea $A:=\{n\in \mathbb{N} : n \leq 20\}$, $B:=\{3n-1 : n\in \mathbb{N}\}$ y $C:=\{2n+1 : n\in \mathbb{N}\}$, describir los conjuntos:
\begin{enumerate}
\item $A\cap B\cap C$
\item $(A\cap B)\setminus C$
\item $(A\cap C)\setminus B$
\end{enumerate}
\end{ej}
\begin{sol}
Cosas de prueba
\begin{enumerate}
\item En primer lugar, si $m\in B \cap A$, entonces nos preguntamos por los elementos de $B$ que son menores o iguales que $20$, es decir:
$$3n-1 \leq 20 \Leftrightarrow \underbrace{n \leq \frac{21}{3} = 7}_{\mbox{\footnotesize{Esto indica que $1\leq n \leq 7$}}} \Rightarrow B\cap A = \underbrace{\{2,5,8,11,14,17,20\}}_{\mbox{\footnotesize{Por eso sólo hay 7}}}$$
Del mismo modo, podemos expresar $A\cap C$ como:
$$2n+1 \leq 20 \Leftrightarrow n \leq \frac{19}{2} \stackrel{n\in \mathbb{N}}{\Rightarrow} \underbrace{n\leq \frac{18}{2} = 9}_{\mbox{\footnotesize{De nuevo, $1\leq n\leq 9$}}} \Rightarrow C\cap A = \{3,5,7,9,11,13,15,17,19\}$$
Luego, $A\cap B\cap C = A\cap B \cap A \cap C$, es decir, los comunes a los $2$ conjuntos anteriores, por tanto:
$$A\cap B \cap C = \{5,11,17\}$$

\item Como ya tenemos calculada $A\cap B$, calcular $(A\cap B)\setminus C$ no es más que quitar los elementos de $C$. Es más, como los elementos de $A\cap B$ son menores que $20$ (por definición) realmente tenemos que quitar los elementos de $C$ menores que $20$ (pues los otros seguro que no están para quitarlos) y, por tanto, tenemos que quitar los elementos de $C\cap A$, luego:
$$(A\cap B)\setminus C = \{2,8,14,20\}$$

\item De nuevo, tenemos calculada $A\cap C$ y hay que quitar los elementos de $B$, pero como los elementos de $A \cap C$ son menores que $20$ sólo hay que quitar los elementos de $B$ menores que $20$ (porque los demás no están) y, por tanto, hay que quitar $B\cap A$:
$$(A\cap C)\setminus B = \{3,7,9,13,15,19\}$$
\end{enumerate}
\end{sol}

\begin{ej}
Mediante diagramas, identifica los siguientes conjuntos:
\end{ej}

\begin{ej}
Sea $I$ un conjunto de índices y para cada $i \in I$ sea $A_i$ un conjunto. Si $B$ es otro conjunto demuestra que:
$$\left(\bigcup_{i\in I} A_i\right) \cap B = \bigcup_{i\in I} (A_i\cap B)$$
\end{ej}
\begin{sol}
En este tipo de ejercicios, para demostrar la igualdad de dos conjuntos se suele demostrar que uno está contenido en el otro y viceversa.
\begin{itemize}
\item $\subset$

Para hacer el contenido de izquierda a derecha, suponemos que cogemos un elemento del de la izquierda y hay que ver que está en la derecha:
$$x\in \left(\bigcup_{i\in I} A_i\right) \cap B \Rightarrow x \in \left(\bigcup_{i\in I} A_i\right) \ \wedge \ x \in B \Rightarrow (\exists i  \in I : x\in A_i) \ \wedge \ x\in B \Rightarrow $$
$$\Rightarrow \exists i \in I : (x\in A_i \ \wedge \ x\in B) \Rightarrow \exists i \in I : (x\in A_i \cap B) \Rightarrow \bigcup_{i\in I}(A_i\cap B)$$

\item $\supset$

Para ver el contenido de derecha a izquierda, volvemos a partir de un elemento de la derecha y tenemos que ver que está en la izquierda. En este caso, se puede coger la demostración anterior empezando por el final, puesto que a pesar de que se han puesto implicaciones, de hecho son equivalencias.
$$x\in \left(\bigcup_{i\in I} A_i\right) \cap B \Leftrightarrow x \in \left(\bigcup_{i\in I} A_i\right) \ \wedge \ x \in B \Leftrightarrow (\exists i  \in I : x\in A_i) \ \wedge \ x\in B \Leftrightarrow $$
$$\Leftrightarrow \exists i \in I : (x\in A_i \ \wedge \ x\in B) \Leftrightarrow \exists i \in I : (x\in A_i \cap B) \Leftrightarrow \bigcup_{i\in I}(A_i\cap B)$$
\end{itemize}
\end{sol}

\end{document}